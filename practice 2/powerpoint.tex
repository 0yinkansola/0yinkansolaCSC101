\documentclass{article} 
\usepackage{graphicx}
\begin{document}
	\textbf{A HISTORICAL VIEW: PROGRAMMING LANGUAGES}\\
	\includegraphics{picture8}
	\newpage
	\textbf{JAVA}\\
	In the year 1991, James Gosling, Patrick Naughton and Mike Sheridan began work on a programming language. Originally, it was intended for use on interactive television, but it was too advanced for the digital cable TV industry at the time.\\
	\includegraphics[width=0.5\linewidth]{picture100}\\
	The language was originally named ‘Oak’, then later renamed ‘Green’. But they eventually settled on the name ‘Java’, the language we all know today.
	It was first released to the public in 1996 by a technology company named Sun Microsystems owned by Oracle Corporation, as Java 1.0. 
	Java quickly became popular because it promised ‘write once, run anywhere’ functionality, and allowed network and file access restrictions. It provided easier ways to do things that would have proven tedious and problematic with the programming languages in existence at the time.\\
	In December of 1998, Java 2 was released as J2SE 1.2. Versions of Java 2 came with multiple configurations for different types of platforms. For example, J2EE was for enterprise applications usually operated in server environments, J2ME featured application programming interfaces made for mobile applications, and J2SE was designed for desktop computers. However, in 2006, Sun Inc. renamed the Java 2 versions as Java EE, Java ME and Java SE, as part of a marketing strategy.\\
	\includegraphics[width=0.5\linewidth]{picture200}\\
	These days, Java is often taught as a first programming language, even though there are now other languages that are preferred to it. It is believed to set a good foundation for the programmers of the future as we take more steps towards the advancement of technology.\\
	\textbf{IDEs for Java include:}
	\begin{itemize}
		\item Eclipse
		\item NetBeans
		\item Dr Java
	\end{itemize}
	\textbf{Programs that have been developed with Java include:}
	\begin{itemize}
		\item NASA World Wind
		\item Wikipedia Search
		\item Minecraft
	\end{itemize}
\textbf{Some programming languages related to Java are:}
\begin{itemize}
	\item C++
	\item Lisp
	\item Smalltalk
\end{itemize}

\textbf{PYTHON}\\

In the 1980s, a man named Guido van Rossum conceived an idea. A programming language that would be the successor to the then popular ABC. In 1989, began implementing his project, which he called Python.\\
\includegraphics[width=0.5\linewidth]{picture300}\\
A second version of the language was released on 16 October 2000, which was called Python 2.0. This new version included many new features, including a support for Unicode.
In December 2008, Python 3.0 was released to the public. However, it wasn’t very backwards compatible.\\
The latest version of Python available now is called Python 3.9..5.\\
\textbf{IDEs for Python include:}\\
\begin{itemize}
	\item PyCharm
	\item IDLE
	\item Spyder
\end{itemize}
\textbf{Programs that have been developed with Python include:}\\
\begin{itemize}
	\item Pinterest
	\item Spotify
	\item Instagram
\end{itemize}
\textbf{Some programming languages related to Python are:}\\
\begin{itemize}
	\item Java
	\item Scala
	\item Anaconda
\end{itemize}
	\textbf{C++}\\
	The programming language known as C++ first appeared in 1983 and was developed by Bjarne Stroustrup. It was based on the language called C. At it’s genesis, it was called ‘C with classes’. It was aimed at making improvements on C, by adding features based on object-oriented programming. Over time, other features were added to the language, differentiating it even more from it’s predecessor, C.\\
	\includegraphics[width=0.5\linewidth]{picture400}\\
	Most of the major websites and apps on the internet today have services written in C++, some examples will be seen on the next slide.\\\includegraphics[width=0.5\linewidth]{picture500}\\
	\textbf{IDEs for C++ include:}
	\begin{itemize}
		\item Code::Blocks
		\item CLion
		\item CodeLite
	\end{itemize}
\textbf{Programs that have been developed with C++ include:}
\begin{itemize}
	\item Youtube
	\item Amazon
	\item Mozilla Firefox
\end{itemize}
\textbf{Some programming languages related to C++ are:}
\begin{itemize}
	\item Python
	\item Ruby
	\item C SHARP
\end{itemize}
\textbf{HTML}
HyperText Markup Language, or HTML, is a programming language used for creating webpages. It was developed by the World Wide Web Consortium in 1993. Their aim was to create a simple programming language that can be used by anyone to create simple webpages.\\
\includegraphics[width=0.5\linewidth]{picture600}\\
HTML uses basic tags that are written in angle brackets to tell the browser what a webpage is made of. Tags usually come in pairs. A pair of tags includes an opening tag and a closing tag. Closing tags always have a forward slash after the opening bracket, to indicate that it is a closing tag.\\
 \textbf{Examples of tags in HTML include:}

\begin{itemize}
	\item <!DOCTYPE>
	\item <html> </html>
	\item <head> </head>
	\item <title> </title>
	\item <body> </body>	
\end{itemize}
The latest version of HTML available now is HTML5.\\
\textbf{IDEs for HTML include:}
\begin{itemize}
	\item Adobe Dreamweaver
	\item WebStorm
	\item PhpStorm
\end{itemize}
\textbf{Programs that have been developed with HTML include:}
\begin{itemize}
	\item Notepad++
	\item Atom
	\item Sublime Text
\end{itemize}
\textbf{Some programming languages related to HTML are:}
\begin{itemize}
	\item XHTML
	\item CSS
	\item Python
\end{itemize}
\textbf{JAVASCRIPT}\\

Despite the similarities in nomenclature, Java and JavaScript are entirely unrelated. JavaScript first appeared in 1995. It was developed by Brendan Eich of Netscape and was originally designed to fix the problem of static webpages after they were loaded in the browser. The language included some features that would solve the problem and JavaScript soon became one of the most used programming languages in the world.
\\
\includegraphics[width=0.5\linewidth]{picture700}\\
\textbf{IDEs for Javascript include:}\\
\begin{itemize}
	\item Web Storm
	\item Atom
	\item Komodo Edit
\end{itemize}
\textbf{Programs that have been developed with Javascript include:}\\
\begin{itemize}
	\item Uber
	\item Candy Crush
	\item Netflix
\end{itemize}
\textbf{Some programming languages related to Javascript are:}\\
\begin{itemize}
	\item Dart
	\item CoffeeScript
	\item TypeScript
\end{itemize}
\end{document}